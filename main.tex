\documentclass{sducv}
% 定义新的命令

% 背景
\newcommand{\resumeHead}[1]{%
    \begin{tikzpicture}[remember picture, overlay]
        % 背景图
        \node[anchor=north, inner sep=0pt](header) at (current page.north){%
            \includegraphics[width=\paperwidth]{images/background_head.pdf}%
        };
        % 左侧图标
        \node[anchor=west](school_logo) at (header.west){%
            \hspace{0.5cm}%
            \includegraphics[width=0.2\textwidth]{images/banner_white.pdf}%
        };
        % 右侧文字
        \node[anchor=east](school_name) at (header.east){%
            \textcolor{white}{\textbf{#1}}%
            \hspace{0.5cm}%
        };
    \end{tikzpicture}
    % 调整间距
    \vspace{2em} % 在此处明确指定长度
}

\newcommand{\resumeBody}{
    \begin{tikzpicture}[remember picture, overlay]
        \node[opacity=0.05] at(current page.center){
            \includegraphics[width=0.7\paperwidth, keepaspectratio]{images/background_body.pdf}
        };
    \end{tikzpicture}
}

\newcommand{\resumefoot}[4]{
       \begin{tikzpicture}[remember picture, overlay]
        \node[anchor=south, inner sep=0pt](footer) at (current page.south){
            \includegraphics[width=\paperwidth]{images/background_foot.pdf}
        };
        % 联系方式
        \node[anchor=center] at(footer.center){
        
        \textcolor{white}{
        % 邮箱
        \faEnvelope \quad \href{mailto:#1}{#1}
        \hspace{2em}
        % 手机号
        \faPhone \quad  #2
        % 别的联系方式,如微信、GitHub等
        \hspace{2em}
        \faGithub \quad \href{mailto:https://github.com/#3}{ #3}
        \hspace{2em}
        \faBlog \quad \href{mailto:https://#4}{ #4}
    }
        };
    \end{tikzpicture}
}

\begin{document} 
% 页眉可以注释
\resumeHead{网络空间安全学院 | School of Cyber Science and Technology   }

\resumeBody
\personalInfoWithPhoto{潮流美海}{男}{2001-04}{中共党员}{山东·青岛}{151-8888-8888}{youremail@mail.sdu.edu.cn}



% 如果想插入照片,请取消此代码的注释。
\begin{textblock*}{0cm}(17cm,2.3cm) % 图片的宽度和位置 (x, y)
    \includegraphics[width=2cm]{images/avatar.png} % 插入图片
\end{textblock*}

\section{教育经历}

\ResumeItem{山东大学}
[\textnormal{网络空间安全学院,网络与信息安全专业|}  硕士研究生]
[2023.09—2026.06]

主要研究方向为\textbf{零知识证明},具有CUDA、Ascend等工程经验。

\ResumeItem{哈尔滨工程大学}
[\textnormal{计算机科学与技术学院,信息安全专业|} 本科]
[2019.09—2023.06]

\textbf{Rank: 2/22(获得推免资格)},获学业奖学金多次,优秀共青团员,优秀班干部

\section{技术能力}
\begin{itemize}
  \item \textbf{语言}: 编程不受特定语言限制。常用 Rust, Golang, Python,C++; 熟悉 C, \GrayText{JavaScript};了解 Lua, Java, \GrayText{TypeScript}。
  \item \textbf{工作流}: Linux, Shell, (Neo)Vim, Git, GitHub, GitLab.
  \item \textbf{其他}: 有容器化技术的实践经验,熟悉 Kubernetes 的使用。
\end{itemize}

\section{实习经历}

\ResumeItem{北京 ABCD 有限公司}
[后端开发实习生/XXXX]
[2020.10—2021.03] 

\begin{itemize}
  \item \textbf{独立负责XXX业务后端的设计、开发、测试和部署。}通过 FaaS、Kafka 等平台实现站内信模板渲染服务。向上游提供 SDK 代码,增加或升级了多种离线和在线逻辑。完成了业务对站内信的多样需求。
  \item \textbf{参与 XXX 的需求分析,系统技术方案设计;完成需求开发、灰度测试、上线和监控。}
\end{itemize}

\section{项目经历}

\ResumeItem{论文}
[ \textnormal{CCF-A}]
[DAC|2025]
\begin{itemize}
  \item \textbf{介绍:}
  \item \textbf{创新:}
  \item \textbf{结果:}
  \item \textbf{收获:}
\end{itemize}

\ResumeItem{项目}
[ \textnormal{个人项目}]
[2023-2024]
\begin{itemize}
  \item \textbf{介绍:}
  \item \textbf{技术:}
  \item \textbf{结果:}
  \item \textbf{收获:}
\end{itemize}



\section{校园经历}
\begin{itemize}
\item 班级组织委员。
\item 党支部委员,负责发展党员。
\end{itemize}
\section{个人总结}

\begin{itemize}
  \item 本人乐观开朗、在校成绩优异、自驱能力强,具有良好的沟通能力和团队合作精神。
\end{itemize}

\resumefoot{yourEmail@mail.sdu.edu.cn}{151-0000-0000}{github}{website}

\end{document}
